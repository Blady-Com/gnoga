%  -*- coding: utf-8 -*-
%
%  ZanyBlue, an Ada library and framework for finite element analysis.
%
%  Copyright (c) 2012, 2016, Michael Rohan <mrohan@zanyblue.com>
%  All rights reserved.
%
%  Redistribution and use in source and binary forms, with or without
%  modification, are permitted provided that the following conditions
%  are met:
%
%    * Redistributions of source code must retain the above copyright
%      notice, this list of conditions and the following disclaimer.
%
%    * Redistributions in binary form must reproduce the above copyright
%      notice, this list of conditions and the following disclaimer in the
%      documentation and/or other materials provided with the distribution.
%
%    * Neither the name of ZanyBlue nor the names of its contributors may
%      be used to endorse or promote products derived from this software
%      without specific prior written permission.
%
%  THIS SOFTWARE IS PROVIDED BY THE COPYRIGHT HOLDERS AND CONTRIBUTORS
%  "AS IS" AND ANY EXPRESS OR IMPLIED WARRANTIES, INCLUDING, BUT NOT
%  LIMITED TO, THE IMPLIED WARRANTIES OF MERCHANTABILITY AND FITNESS FOR
%  A PARTICULAR PURPOSE ARE DISCLAIMED. IN NO EVENT SHALL THE COPYRIGHT
%  HOLDER OR CONTRIBUTORS BE LIABLE FOR ANY DIRECT, INDIRECT, INCIDENTAL,
%  SPECIAL, EXEMPLARY, OR CONSEQUENTIAL DAMAGES (INCLUDING, BUT NOT LIMITED
%  TO, PROCUREMENT OF SUBSTITUTE GOODS OR SERVICES; LOSS OF USE, DATA, OR
%  PROFITS; OR BUSINESS INTERRUPTION) HOWEVER CAUSED AND ON ANY THEORY OF
%  LIABILITY, WHETHER IN CONTRACT, STRICT LIABILITY, OR TORT (INCLUDING
%  NEGLIGENCE OR OTHERWISE) ARISING IN ANY WAY OUT OF THE USE OF THIS
%  SOFTWARE, EVEN IF ADVISED OF THE POSSIBILITY OF SUCH DAMAGE.
%

\chapter{Introduction}

This manual describes the use of the ZanyBlue libraries, an Ada framework for
finite element analysis with supporting libraries which might be of general
use in other projects.  The initial implementation contains the \texttt{Text}
package (and supporting packages and utilities) supplying functionality
mirroring the Java \texttt{MessageFormat} package.  The \texttt{Text}
functionality allows the application messages to be externalized in
\texttt{.properties} files and localized into additional languages.  As a
major side effect the formatting of arguments within messages is needed and
implemented.

This, naturally, led to a need for a testing framework.  AUnit is used for
unit testing of the libraries, however, testing of the command line utilities
proved more difficult.  While DejaGNU was available, it seemed a little more
involved than was needed for this project.  A new testing application was
written, \texttt{zbtest}, which is a hierarchical testing framework supporting
definition scopes.

\begin{htmlonly}
This document is available as a
\htmladdnormallink{single PDF file}{../zanyblue.pdf}.

Package documentation is available via
\htmladdnormallink{GNAT HTML}{ref/index.html} generated file.
\end{htmlonly}

\section{Building the Library and Applications}

Building the ZanyBlue library and applications does not require a configure
step (at this time).  The build requirements are similar on Unix (Linux)
and Windows (the Makefiles use the environment variable \verb|OS| to determine
whether the build is Linux (the variable is not defined) or Windows (where it
is defined with the value \verb|Windows_NT|).

\subsection{Requirements}

The build requires
\begin{itemize}
\item A recent version of GNAT (e.g., GPL 2010, 2011) from AdaCore.
\item GNU make.
\item The XMLAda package needs to be installed to build the ZanyBlue
      testing application, \texttt{zbtest}.
\item The unit tests require the AUnit package, e.g., GPL 2011 version.
\end{itemize}

\subsection{Building}

Once the required environment is in place, building is simply a matter of
issuing ``\texttt{make}'' in the ``\texttt{src}'' directory, e.g.,
\begin{small}
\begin{verbatim}
    $ make -C src
\end{verbatim}
\end{small}
This produces the library in the ``\texttt{lib/zanyblue}'' directory,
and the two executables ``\texttt{zbmcompile}'', the message compiler,
and ``\texttt{zbtest}'' the ZanyBlue testing application.  The ZanyBlue
package spec files and generic bodies are copied to the
``\texttt{include/zanyblue}'' directory.

\subsection{Using with GNAT}

A ZanyBlue GNAT \texttt{gprbuild} project file is available in the
directory ``\texttt{lib/gnat}'' directory.  Adding this directory to
your ``\verb|ADA_PROJECT_PATH|'' should allow the use of the ZanyBlue
library with \texttt{gprbuild} projects.  See the text examples and
the GPS build instructions.

\subsection{Testing}

Testing is via the ``\texttt{check}'' target.  From the ``\texttt{src}''
directory this will run both the unit and system tests.  If experimenting,
running these tests independently is probably more useful (the combined
testing is normally only run from under Jenkins where the summary XML files
are loaded as the test results for the build):
\begin{small}
\begin{verbatim}
    $ make -C src/test/unittest check
    $ make -C src/test/system check
\end{verbatim}
\end{small}

\subsection{Examples}

The examples include a Gtk example.  If Gtk is not installed, this example
will fail to build (but the other should be OK).  The ``\texttt{dumplocale}''
example is useful when examining the built-in locale data, e.g., to see
the date, time, numeric, etc data for Swedish,
\begin{small}
\begin{verbatim}
    $ make -C examples/text/dumplocale
    $ make -C examples/text/dumplocale run_sv
\end{verbatim}
\end{small}

\subsection{Windows Issues}

The installation of the Windows GNU Win32 version of make,
\url{http://gnuwin32.sourceforge.net/packages/make.htm}, does not update
the Path environment. This should be done manually after installing
via the Control Panel $\rightarrow$ System $\rightarrow$ Advanced
$\rightarrow$ Environment Variables $\rightarrow$ Path and adding the
path \verb|C:\Program Files\GnuWin32\bin|.

The majority of the example application are text based. This is
problematic on Windows systems as the standard Windows command console
does not understand Unicode output. This makes the Gtk example application
the only fully functional example application on Windows systems. To
build this application, the GtkAda package should be installed, again from
AdaCore and, as for AUnit, the gprbuild project path environment variable
will need to be set if GtkAda is not installed in the default location.
